
%% bare_conf.tex
%% V1.4b
%% 2015/08/26
%% by Michael Shell
%% See:
%% http://www.michaelshell.org/
%% for current contact information.
%%
%% This is a skeleton file demonstrating the use of IEEEtran.cls
%% (requires IEEEtran.cls version 1.8b or later) with an IEEE
%% conference paper.
%%
%% Support sites:
%% http://www.michaelshell.org/tex/ieeetran/
%% http://www.ctan.org/pkg/ieeetran
%% and
%% http://www.ieee.org/

%%*************************************************************************
%% Legal Notice:
%% This code is offered as-is without any warranty either expressed or
%% implied; without even the implied warranty of MERCHANTABILITY or
%% FITNESS FOR A PARTICULAR PURPOSE! 
%% User assumes all risk.
%% In no event shall the IEEE or any contributor to this code be liable for
%% any damages or losses, including, but not limited to, incidental,
%% consequential, or any other damages, resulting from the use or misuse
%% of any information contained here.
%%
%% All comments are the opinions of their respective authors and are not
%% necessarily endorsed by the IEEE.
%%
%% This work is distributed under the LaTeX Project Public License (LPPL)
%% ( http://www.latex-project.org/ ) version 1.3, and may be freely used,
%% distributed and modified. A copy of the LPPL, version 1.3, is included
%% in the base LaTeX documentation of all distributions of LaTeX released
%% 2003/12/01 or later.
%% Retain all contribution notices and credits.
%% ** Modified files should be clearly indicated as such, including  **
%% ** renaming them and changing author support contact information. **
%%*************************************************************************


% *** Authors should verify (and, if needed, correct) their LaTeX system  ***
% *** with the testflow diagnostic prior to trusting their LaTeX platform ***
% *** with production work. The IEEE's font choices and paper sizes can   ***
% *** trigger bugs that do not appear when using other class files.       ***                          ***
% The testflow support page is at:
% http://www.michaelshell.org/tex/testflow/



\documentclass[conference]{IEEEtran}
% Some Computer Society conferences also require the compsoc mode option,
% but others use the standard conference format.
%
% If IEEEtran.cls has not been installed into the LaTeX system files,
% manually specify the path to it like:
% \documentclass[conference]{../sty/IEEEtran}





% Some very useful LaTeX packages include:
% (uncomment the ones you want to load)


% *** MISC UTILITY PACKAGES ***
%
%\usepackage{ifpdf}
% Heiko Oberdiek's ifpdf.sty is very useful if you need conditional
% compilation based on whether the output is pdf or dvi.
% usage:
% \ifpdf
%   % pdf code
% \else
%   % dvi code
% \fi
% The latest version of ifpdf.sty can be obtained from:
% http://www.ctan.org/pkg/ifpdf
% Also, note that IEEEtran.cls V1.7 and later provides a builtin
% \ifCLASSINFOpdf conditional that works the same way.
% When switching from latex to pdflatex and vice-versa, the compiler may
% have to be run twice to clear warning/error messages.






% *** CITATION PACKAGES ***
%
%\usepackage{cite}
% cite.sty was written by Donald Arseneau
% V1.6 and later of IEEEtran pre-defines the format of the cite.sty package
% \cite{} output to follow that of the IEEE. Loading the cite package will
% result in citation numbers being automatically sorted and properly
% "compressed/ranged". e.g., [1], [9], [2], [7], [5], [6] without using
% cite.sty will become [1], [2], [5]--[7], [9] using cite.sty. cite.sty's
% \cite will automatically add leading space, if needed. Use cite.sty's
% noadjust option (cite.sty V3.8 and later) if you want to turn this off
% such as if a citation ever needs to be enclosed in parenthesis.
% cite.sty is already installed on most LaTeX systems. Be sure and use
% version 5.0 (2009-03-20) and later if using hyperref.sty.
% The latest version can be obtained at:
% http://www.ctan.org/pkg/cite
% The documentation is contained in the cite.sty file itself.






% *** GRAPHICS RELATED PACKAGES ***
%
\ifCLASSINFOpdf
% \usepackage[pdftex]{graphicx}
  % declare the path(s) where your graphic files are
  % \graphicspath{{../pdf/}{../jpeg/}}
  % and their extensions so you won't have to specify these with
  % every instance of \includegraphics
  % \DeclareGraphicsExtensions{.pdf,.jpeg,.png}
\else
  % or other class option (dvipsone, dvipdf, if not using dvips). graphicx
  % will default to the driver specified in the system graphics.cfg if no
  % driver is specified.
  % \usepackage[dvips]{graphicx}
  % declare the path(s) where your graphic files are
  % \graphicspath{{../eps/}}
  % and their extensions so you won't have to specify these with
  % every instance of \includegraphics
  % \DeclareGraphicsExtensions{.eps}
\fi
% graphicx was written by David Carlisle and Sebastian Rahtz. It is
% required if you want graphics, photos, etc. graphicx.sty is already
% installed on most LaTeX systems. The latest version and documentation
% can be obtained at: 
% http://www.ctan.org/pkg/graphicx
% Another good source of documentation is "Using Imported Graphics in
% LaTeX2e" by Keith Reckdahl which can be found at:
% http://www.ctan.org/pkg/epslatex
%
% latex, and pdflatex in dvi mode, support graphics in encapsulated
% postscript (.eps) format. pdflatex in pdf mode supports graphics
% in .pdf, .jpeg, .png and .mps (metapost) formats. Users should ensure
% that all non-photo figures use a vector format (.eps, .pdf, .mps) and
% not a bitmapped formats (.jpeg, .png). The IEEE frowns on bitmapped formats
% which can result in "jaggedy"/blurry rendering of lines and letters as
% well as large increases in file sizes.
%
% You can find documentation about the pdfTeX application at:
% http://www.tug.org/applications/pdftex





% *** MATH PACKAGES ***
%
%\usepackage{amsmath}
% A popular package from the American Mathematical Society that provides
% many useful and powerful commands for dealing with mathematics.
%
% Note that the amsmath package sets \interdisplaylinepenalty to 10000
% thus preventing page breaks from occurring within multiline equations. Use:
%\interdisplaylinepenalty=2500
% after loading amsmath to restore such page breaks as IEEEtran.cls normally
% does. amsmath.sty is already installed on most LaTeX systems. The latest
% version and documentation can be obtained at:
% http://www.ctan.org/pkg/amsmath





% *** SPECIALIZED LIST PACKAGES ***
%
%\usepackage{algorithmic}
% algorithmic.sty was written by Peter Williams and Rogerio Brito.
% This package provides an algorithmic environment fo describing algorithms.
% You can use the algorithmic environment in-text or within a figure
% environment to provide for a floating algorithm. Do NOT use the algorithm
% floating environment provided by algorithm.sty (by the same authors) or
% algorithm2e.sty (by Christophe Fiorio) as the IEEE does not use dedicated
% algorithm float types and packages that provide these will not provide
% correct IEEE style captions. The latest version and documentation of
% algorithmic.sty can be obtained at:
% http://www.ctan.org/pkg/algorithms
% Also of interest may be the (relatively newer and more customizable)
% algorithmicx.sty package by Szasz Janos:
% http://www.ctan.org/pkg/algorithmicx




% *** ALIGNMENT PACKAGES ***
%
%\usepackage{array}
% Frank Mittelbach's and David Carlisle's array.sty patches and improves
% the standard LaTeX2e array and tabular environments to provide better
% appearance and additional user controls. As the default LaTeX2e table
% generation code is lacking to the point of almost being broken with
% respect to the quality of the end results, all users are strongly
% advised to use an enhanced (at the very least that provided by array.sty)
% set of table tools. array.sty is already installed on most systems. The
% latest version and documentation can be obtained at:
% http://www.ctan.org/pkg/array


% IEEEtran contains the IEEEeqnarray family of commands that can be used to
% generate multiline equations as well as matrices, tables, etc., of high
% quality.




% *** SUBFIGURE PACKAGES ***
\usepackage{verbatimbox}
\ifCLASSOPTIONcompsoc
  \usepackage[caption=false,font=normalsize,labelfont=sf,textfont=sf]{subfig}
\else
  \usepackage[caption=false,font=footnotesize]{subfig}
\fi
% subfig.sty, written by Steven Douglas Cochran, is the modern replacement
% for subfigure.sty, the latter of which is no longer maintained and is
% incompatible with some LaTeX packages including fixltx2e. However,
% subfig.sty requires and automatically loads Axel Sommerfeldt's caption.sty
% which will override IEEEtran.cls' handling of captions and this will result
% in non-IEEE style figure/table captions. To prevent this problem, be sure
% and invoke subfig.sty's "caption=false" package option (available since
% subfig.sty version 1.3, 2005/06/28) as this is will preserve IEEEtran.cls
% handling of captions.
% Note that the Computer Society format requires a larger sans serif font
% than the serif footnote size font used in traditional IEEE formatting
% and thus the need to invoke different subfig.sty package options depending
% on whether compsoc mode has been enabled.
%
% The latest version and documentation of subfig.sty can be obtained at:
% http://www.ctan.org/pkg/subfig




% *** FLOAT PACKAGES ***
%
%\usepackage{fixltx2e}
% fixltx2e, the successor to the earlier fix2col.sty, was written by
% Frank Mittelbach and David Carlisle. This package corrects a few problems
% in the LaTeX2e kernel, the most notable of which is that in current
% LaTeX2e releases, the ordering of single and double column floats is not
% guaranteed to be preserved. Thus, an unpatched LaTeX2e can allow a
% single column figure to be placed prior to an earlier double column
% figure.
% Be aware that LaTeX2e kernels dated 2015 and later have fixltx2e.sty's
% corrections already built into the system in which case a warning will
% be issued if an attempt is made to load fixltx2e.sty as it is no longer
% needed.
% The latest version and documentation can be found at:
% http://www.ctan.org/pkg/fixltx2e


%\usepackage{stfloats}
% stfloats.sty was written by Sigitas Tolusis. This package gives LaTeX2e
% the ability to do double column floats at the bottom of the page as well
% as the top. (e.g., "\begin{figure*}[!b]" is not normally possible in
% LaTeX2e). It also provides a command:
%\fnbelowfloat
% to enable the placement of footnotes below bottom floats (the standard
% LaTeX2e kernel puts them above bottom floats). This is an invasive package
% which rewrites many portions of the LaTeX2e float routines. It may not work
% with other packages that modify the LaTeX2e float routines. The latest
% version and documentation can be obtained at:
% http://www.ctan.org/pkg/stfloats
% Do not use the stfloats baselinefloat ability as the IEEE does not allow
% \baselineskip to stretch. Authors submitting work to the IEEE should note
% that the IEEE rarely uses double column equations and that authors should try
% to avoid such use. Do not be tempted to use the cuted.sty or midfloat.sty
% packages (also by Sigitas Tolusis) as the IEEE does not format its papers in
% such ways.
% Do not attempt to use stfloats with fixltx2e as they are incompatible.
% Instead, use Morten Hogholm'a dblfloatfix which combines the features
% of both fixltx2e and stfloats:
%
% \usepackage{dblfloatfix}
% The latest version can be found at:
% http://www.ctan.org/pkg/dblfloatfix




% *** PDF, URL AND HYPERLINK PACKAGES ***
%
%\usepackage{url}
% url.sty was written by Donald Arseneau. It provides better support for
% handling and breaking URLs. url.sty is already installed on most LaTeX
% systems. The latest version and documentation can be obtained at:
% http://www.ctan.org/pkg/url
% Basically, \url{my_url_here}.

\usepackage[T1]{fontenc}
\usepackage[utf8]{inputenc}
\usepackage{color}
\usepackage[english]{babel}
\usepackage{graphicx}
\usepackage{makecell}
\usepackage{enumitem}
\usepackage{hyperref}
\usepackage{float}
\usepackage{mathtools}
\usepackage{algorithm}
\usepackage{algorithmic}
\renewcommand{\algorithmicrequire}{\textbf{Input :}}
\renewcommand{\algorithmicensure}{\textbf{Output :}}
\usepackage[centering]{geometry}



% *** Do not adjust lengths that control margins, column widths, etc. ***
% *** Do not use packages that alter fonts (such as pslatex).         ***
% There should be no need to do such things with IEEEtran.cls V1.6 and later.
% (Unless specifically asked to do so by the journal or conference you plan
% to submit to, of course. )


% correct bad hyphenation here
\hyphenation{op-tical net-works semi-conduc-tor}


\begin{document}
%
% paper title
% Titles are generally capitalized except for words such as a, an, and, as,
% at, but, by, for, in, nor, of, on, or, the, to and up, which are usually
% not capitalized unless they are the first or last word of the title.
% Linebreaks \\ can be used within to get better formatting as desired.
% Do not put math or special symbols in the title.
\title{Improving quality of an existing Cloud framework for Security issues in data security model}


% author names and affiliations
% use a multiple column layout for up to three different
% affiliations
\author{\IEEEauthorblockN{Chandrahas Raju}
\IEEEauthorblockA{ 940302-3774\\ chara16@student.bth.se}
\and
\IEEEauthorblockN{Madhukar Enugurthi}
\IEEEauthorblockA{950607-7537 \\maen16@student.bth.se}}

% conference papers do not typically use \thanks and this command
% is locked out in conference mode. If really needed, such as for
% the acknowledgment of grants, issue a \IEEEoverridecommandlockouts
% after \documentclass

% for over three affiliations, or if they all won't fit within the width
% of the page, use this alternative format:
% 
%\author{\IEEEauthorblockN{Michael Shell\IEEEauthorrefmark{1},
%Homer Simpson\IEEEauthorrefmark{2},
%James Kirk\IEEEauthorrefmark{3}, 
%Montgomery Scott\IEEEauthorrefmark{3} and
%Eldon Tyrell\IEEEauthorrefmark{4}}
%\IEEEauthorblockA{\IEEEauthorrefmark{1}School of Electrical and Computer Engineering\\
%Georgia Institute of Technology,
%Atlanta, Georgia 30332--0250\\ Email: see http://www.michaelshell.org/contact.html}
%\IEEEauthorblockA{\IEEEauthorrefmark{2}Twentieth Century Fox, Springfield, USA\\
%Email: homer@thesimpsons.com}
%\IEEEauthorblockA{\IEEEauthorrefmark{3}Starfleet Academy, San Francisco, California 96678-2391\\
%Telephone: (800) 555--1212, Fax: (888) 555--1212}
%\IEEEauthorblockA{\IEEEauthorrefmark{4}Tyrell Inc., 123 Replicant Street, Los Angeles, California 90210--4321}}




% use for special paper notices
%\IEEEspecialpapernotice{(Invited Paper)}




% make the title area
\maketitle

% As a general rule, do not put math, special symbols or citations
% in the abstract
\begin{abstract}
    Cloud Computing has become an area of importance with increased applications in many software companies by offering many benefits in terms of low cost and data accessibility. However, there are still many challenges in solving these security issues in handling data. Therefore, security issues still remain the primary inhibitor to adoption of Cloud Computing services.  The main objective of our research is to  improve  the quality of an existing framework in data security model. In this article, we strive to achieve the above goals with the help of a survey.
On conducting a survey, We identified the hypothesis behind these issues and provide support for future research on Data security issue in cloud computing.
\end{abstract}

% no keywords




% For peer review papers, you can put extra information on the cover
% page as needed:
% \ifCLASSOPTIONpeerreview
% \begin{center} \bfseries EDICS Category: 3-BBND \end{center}
% \fi
%
% For peerreview papers, this IEEEtran command inserts a page break and
% creates the second title. It will be ignored for other modes.
\IEEEpeerreviewmaketitle




\section{Group members participation}

The group members participated in the idea creation and report writing with the amount of involvment displayed in table \ref{table:participation}.

\begin{table}[ht!]
\centering
\addvbuffer[8pt 8pt]{\begin{tabular}{|c|c|c|}
  \hline
  \textbf{Group member}&\textbf{Idea creation}&\textbf{Report writing}\\
  \hline
  \makecell{Chandrahas\\ Raju }&50\%&50\%\\
  \hline
  \makecell{Madhukar\\ Enugurthi}&50\%&50\%\\
  \hline 
\end{tabular}}
\caption{Work repartition}
\label{table:participation}
\end{table}


\section{Introduction}

Cloud Computing is one of the best emerging parameter in IT industry i.e. Cloud Computing is the latest trend in IT market. Users access their files and applications by accessing internet through cloud storage. This enable the user to utilize the set of resources of the server over the network. Cloud involves accessing the software applications. Cloud Computing also provides data storage and processing power over the Internet. Cloud Computing is not just a third party data warehouse. The data stored in the cloud may be frequently updated by the users, including insertion, deletion, modification, appending, reordering, etc.
The term cloud has been used historically as a metaphor for the Internet. This usage was originally Derived from its common depiction in network diagrams as an outline of a cloud, used to represent the transport of data across carrier backbones (which owned the cloud) to an endpoint location on the other side of the cloud.
There are three types of service layered models in Cloud Computing. Software as a service(SaaS), Platform as a Service(PaaS) and Infrastructure as a Service(IaaS). Customer obtain the facility to access & use an application or service that is hosted in the cloud for example Microsoft provides software as a service in Microsoft Office web application are accessible to office volume licensing customer and office web application is called SaaS. Customer obtain access to the platform by enabling them to organize their own software and application on cloud environment and control their application but they can’t manage servers and storage. It shares platform for customer software application configuration, testing and development of application this is called PaaS. The facility provided to the customer is to leave processing storage and other fundamental computing resources. The customer does not manage or control the basic cloud infrastructure but has control over Operating System, storage, deployed application is called IaaS.


\subsection{Background}

The main concern in Cloud Computing is the security issue, even though many researchers have been focused in this area they have not found the perfect solution yet. The impact of cloud services on the business sector is tremendous. With the increase in the end-users, there is an increasing growth in the number of Cloud Service Providers (CSP) as well. The CSP is a third party that maintains and manages information about another entity [3]. To know the security aspects and frameworks that are mitigating the issue which are present in current literature, we studied few Systematic Literature Review (SLR) and articles about security issues to gain knowledge in the present literature
As we studied about few SLR’s and articles regarding security issues in Cloud Computing, we gained some knowledge about what are the issues Cloud Computing is facing in security. By analyzing all the SLR’s we have found that the effective framework is missing to mitigate the data security model issues. So our purpose of this research is to provide improvement in quality of existing framework related to data security in Cloud Computing. 

\subsection{Objectives}

The main aim of these research is to improve the quality of the data security in cloud. That is how the data is going to be secured from malicious prosecution.
The main focus of this research is to make the readers gain the knowledge about quality improvement in  data security.

\subsection{Structure} 
The report is structured in the following way:  
Section III gives the motivation and background of this report. Section IV shows how the plan in conducting the research is done. Section V gives deals with how the research have been performed. Data extraction, analysis of data and interpretation of the results were discussed in Section VI. Previous knowledge with important novelties and their threats to validity in research is discussed in Section VII. Conclusion and summary of our work is explained in the final or last section.   


\nocite{*}


\section{Background and Motivation}

The focus on cloud computing was a obtained when the search was made related to latest technologies. Many of the results shown to us was related to Cloud Computing and effects of security in cloud. Later we gained some knowledge in Cloud Computing and found that security was the major problem in the cloud then our focus was shifted to how to save the data in cloud. Then we have studied about the security models in the Cloud Computing.

\paragraph{}
Initially a reviews was assist to know importance of am ligating Cloud and security, then later we have planned to target on improvement in the security. Cloud is having a feature called parallel computing which allows to store lager amount of data. Cloud is basically on the remote servers which can have large measure of data for which the security issues are more.

\paragraph{}
 The Author of (Data security issues) has conducted a survey on problems that the cloud is facing in the security and privacy which helps the future developers in building a better security system.
The previous Authors said that use of cloud has rapidly increased, even though it has a wide range of market there is a lack in providing the security. Customers does not want to lose their private information as a result of malicious insider in the cloud. The purpose of this survey is to research on the security issues to eradicated the data lose in Cloud.


\section{Research Definition}

\subsection{Research Objectives}
The main objectives of these research is to provide improvements in the quality to an existing framework in data security. To identify basic structure for the framework related to data security in Cloud. Further analyzing of data in literature review for identifying the existing framework related to data security. 

\subsection{Research Questions}
 The research questions are:
\begin{itemize}
\item{RQ1} What is the state of art for the framework related to data security?

\item{RQ2} How can the selected framework will increase the quality to mitigate the security issues relates to data security model?

\item{RQ3} What are the mitigation factors faced by the cloud regarding data security?
\end{itemize}

\subsection{Research Method}
 When we have referred to selected empirical method using preliminary guidelines Kitchenham the ambiguity in selection of research method was solved. The above reference which we have given consist information related to selected empirical research method. There are five methods of research according to them which is relevant to software engineering are research by survey, case study, controlled experiments, action research and ethnography.   
\paragraph{}
The study on security models in cloud computing helps us answering RQ1.
 RQ2 and RQ3 are answered by conducting literature review to gain more knowledge about security issues in cloud. Data extraction from literature review is analyzed and we gain knowledge about framework related to Data security, by this we can answer RQ2 and further we provide enhancement to an existing framework which mitigates Data security model in Cloud Computing. On providing these improvements in quality to an existing framework we can answer RQ3.
\paragraph{}     
To answer all these research questions, we have selected quantitative method for data collection. The quantitative method is practitioner survey. We have selected this survey based on our research questions. This research questions can be answered by our targeted audience. The data collection is analyzed using grounded theory approach and that data is used to answer our research questions.

\subsection{Unit of Analysis}
The unit of analysis we have chosen after the research method was audience who are professional in our particular subject/topic. We have prepared a questionnaire and it is kept openly in the LinkedIn site. All the individuals who are answering that questionnaire are selected. After the selection, judgment sampling is done to those selected people. We have got few responses for our questionnaire. By this judgment sampling out RQ’s got answered.

\subsection{Data Collection}
We have conducted an online survey through a questionnaire where the data is collected. In our questionnaire there are questions which are taken from our research questions.
We have selected our responses from the highly professional people. By asking them the questions like their age of experience, where did they train in that particular topic etc. We got few responses and selected few responses which we felt as the best ones.

\subsection{Data analysis}
After the data extraction process is done from the resources, we started analyzing that data. We have divided the work between our team mates. We have chosen partitioning method sampling method. The data collected from different users are partitioned according to their age, experience and all. Then we have divide our work and analyzed the data individually and peer reviewed others work after the analysis and concluded the answers.


\section{Research Operation}
We started our research with a team of two people. We searched for what are the issues that Cloud Computing is facing. By searching this we have found that security is the major issue that the cloud has. A systematic literature review has been conducted for our topic in which we have referred to many of the papers related to our topic and found some gaps in that references. After finding the gaps in that research we have formulated the research questions as seen in the research definition and planning section (Research questions). The research questions are analyzed to select method for conducting a survey. Analysis is done by taking all the responses into consideration. The responses were taken into consideration in such a way that the people who are professional in Cloud Computing. The whole research process took 29 days.
\paragraph{}
After the targeted professionals are identified we have prepared a questionnaire using google forms. The questions were related data security issues in cloud. It took 15days to get all the responses from the professionals. All the responses are compared and drawn a conclusion from it.


\subsection{Quality Assurance}

While doing an online survey there is a risk of user’s bias. To overcome such kind of things we have framed questionnaire in such a way that user can answer questions without any preferences. All the answers came from the audience may not be correct, this may effect the quality of the research. To over come such kind of quality problems results from research websites are given more preference than the direct answers.

\section{Data Analysis and Interpretation}
\subsection{Data Analysis}
The data analysis is discussed in this section based on our survey and literature review. Questionnaire is used in finding the data. Totally we got 14 responses from the different professionals. In all the 14 responses we found 5 responses suits our research questions.
\paragraph{}
The response all we got are from the people who are working on cloud computing, software professionals, senior program managers, developers.
\begin{itemize}
\item{RQ1 Solution:} The different set of solutions for secure storing of the data in Cloud Computing are through homomorphic encryption, multi party computation and anonymity. 
\item{RQ2 Solution:} Data centric is one of the way in increasing the security in the Cloud Computing. If the data has gone into a public cloud, data security and governance control is transferred in a whole cloud provider. The best way of protecting the data in Cloud is by keeping the sensitive data secure using the data centric and file level encryption which works for both private and public Cloud Computing environment.
\item{RQ3 Solution:} The mitigating factors faced by the Cloud during the data security is water hole attack, compliance with data privacy laws in multiple geographies and so on. Coming to water hole attack done to Cloud during the security, this attack has 3 stages firstly the attacker does some reconnaissance and research on its target, in which employees of the targeted company who visits regularly. The second stage is insert an exploit into trusted sites. And finally takes advantage of their system vulnerabilities. The other method is Cloud environment is key and you must understand the respective data storage regulations. Most regulations include EU’s very restrictive regulations accept that this a good solution. These are the mitigating factors faced by the cloud during the data security.
\end{itemize}

\section{Discussion}

\subsection{Contribution}
This research provides to explore the improvement in security in cloud computing. Initially our focus was on Cloud Computing after gaining detailed knowledge in Cloud Computing we have found that the security is the major drawback in the cloud, so we have decided to work on the particular area that is about data security issue in cloud. So we have started our research on this topic. After the survey has done, different professionals have given their opinion on our questions we asked. RQ1 is based on the state of art framework related to data security, Our RQ2 is about the improvement in the quality of the data security and finally our RQ3 is about what is the mitigating factor facing regarding the data security by cloud.  

\subsection{Threats to validity}

\begin{itemize}

\item{Construct validity:} In our questionnaire the respondents may misinterpret some terms so this causes construct validity.
To avoid this, we kept our questionnaire simple.
\item{Internal validity:} There is a problem of drawing a wrong interface from the data collection this lead to chance in internal validity.
To avoid this, we used multiple methods of drawing the answers.
\item{External validity:} As we kept a survey on this there is a problem of getting less number of response this cause external validity threat.
To avoid this, we just increased the targeted population
\item{Conclusion validity:} Our conclusion may cause the conclusion validity due to ambiguous drawing in the answers.

\\To avoid this, we have spent enough time to draw the conclusions. 
\\

\section{Summary and Conclusions}

Cloud storage is one of the popular services provided by the cloud providers to store the customer data in a remote server. Even though the Cloud providers advertise that the stored information will be secure and intact, there are security attacks which lead to loss of data. To overcome the loss of data, the data security principles are implemented in different ways to protect the data [5]. It is a vital to take security and privacy into account when designing and using cloud computing services [6].
\paragraph{}
After this research we can conclude that improving the security issue in data can be done by data security method. Our results show the data security can be eradicated using the data centric protection. In further we can eradicate this type of security issues by CIA model also.
\paragraph{}
Due to limited amount of time we have collected only limited data that is only from 3 companies. In the future research we are planning to conduct more research on the methods which we are choosing.



% trigger a \newpage just before the given reference
% number - used to balance the columns on the last page
% adjust value as needed - may need to be readjusted if
% the document is modified later
%\IEEEtriggeratref{8}
% The "triggered" command can be changed if desired:
%\IEEEtriggercmd{\enlargethispage{-5in}}

% references section

% can use a bibliography generated by BibTeX as a .bbl file
% BibTeX documentation can be easily obtained at:
% http://mirror.ctan.org/biblio/bibtex/contrib/doc/
% The IEEEtran BibTeX style support page is at:
% http://www.michaelshell.org/tex/ieeetran/bibtex/
\newpage
\bibliographystyle{unsrt}
\bibliography{bibliography}
%
% <OR> manually copy in the resultant .bbl file
% set second argument of \begin to the number of references
% (used to reserve space for the reference number labels box)

% that's all folks
\end{document}


