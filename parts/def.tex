\section{Research Definition}

\subsection{Research Objectives}
The main objectives of these research is to provide improvements in the quality to an existing framework in data security. To identify basic structure for the framework related to data security in Cloud. Further analyzing of data in literature review for identifying the existing framework related to data security. 

\subsection{Research Questions}
 The research questions are:
\begin{itemize}
\item{RQ1} What is the state of art for the framework related to data security?

\item{RQ2} How can the selected framework will increase the quality to mitigate the security issues relates to data security model?

\item{RQ3} What are the mitigation factors faced by the cloud regarding data security?
\end{itemize}

\subsection{Research Method}
 When we have referred to selected empirical method using preliminary guidelines Kitchenham the ambiguity in selection of research method was solved. The above reference which we have given consist information related to selected empirical research method. There are five methods of research according to them which is relevant to software engineering are research by survey, case study, controlled experiments, action research and ethnography.   
\paragraph{}
The study on security models in cloud computing helps us answering RQ1.
 RQ2 and RQ3 are answered by conducting literature review to gain more knowledge about security issues in cloud. Data extraction from literature review is analyzed and we gain knowledge about framework related to Data security, by this we can answer RQ2 and further we provide enhancement to an existing framework which mitigates Data security model in Cloud Computing. On providing these improvements in quality to an existing framework we can answer RQ3.
\paragraph{}     
To answer all these research questions, we have selected quantitative method for data collection. The quantitative method is practitioner survey. We have selected this survey based on our research questions. This research questions can be answered by our targeted audience. The data collection is analyzed using grounded theory approach and that data is used to answer our research questions.

\subsection{Unit of Analysis}
The unit of analysis we have chosen after the research method was audience who are professional in our particular subject/topic. We have prepared a questionnaire and it is kept openly in the LinkedIn site. All the individuals who are answering that questionnaire are selected. After the selection, judgment sampling is done to those selected people. We have got few responses for our questionnaire. By this judgment sampling out RQ’s got answered.

\subsection{Data Collection}
We have conducted an online survey through a questionnaire where the data is collected. In our questionnaire there are questions which are taken from our research questions.
We have selected our responses from the highly professional people. By asking them the questions like their age of experience, where did they train in that particular topic etc. We got few responses and selected few responses which we felt as the best ones.

\subsection{Data analysis}
After the data extraction process is done from the resources, we started analyzing that data. We have divided the work between our team mates. We have chosen partitioning method sampling method. The data collected from different users are partitioned according to their age, experience and all. Then we have divide our work and analyzed the data individually and peer reviewed others work after the analysis and concluded the answers.


\section{Research Operation}
We started our research with a team of two people. We searched for what are the issues that Cloud Computing is facing. By searching this we have found that security is the major issue that the cloud has. A systematic literature review has been conducted for our topic in which we have referred to many of the papers related to our topic and found some gaps in that references. After finding the gaps in that research we have formulated the research questions as seen in the research definition and planning section (Research questions). The research questions are analyzed to select method for conducting a survey. Analysis is done by taking all the responses into consideration. The responses were taken into consideration in such a way that the people who are professional in Cloud Computing. The whole research process took 29 days.
\paragraph{}
After the targeted professionals are identified we have prepared a questionnaire using google forms. The questions were related data security issues in cloud. It took 15days to get all the responses from the professionals. All the responses are compared and drawn a conclusion from it.


\subsection{Quality Assurance}

While doing an online survey there is a risk of user’s bias. To overcome such kind of things we have framed questionnaire in such a way that user can answer questions without any preferences. All the answers came from the audience may not be correct, this may effect the quality of the research. To over come such kind of quality problems results from research websites are given more preference than the direct answers.

\section{Data Analysis and Interpretation}
\subsection{Data Analysis}
The data analysis is discussed in this section based on our survey and literature review. Questionnaire is used in finding the data. Totally we got 14 responses from the different professionals. In all the 14 responses we found 5 responses suits our research questions.
\paragraph{}
The response all we got are from the people who are working on cloud computing, software professionals, senior program managers, developers.
\begin{itemize}
\item{RQ1 Solution:} The different set of solutions for secure storing of the data in Cloud Computing are through homomorphic encryption, multi party computation and anonymity. 
\item{RQ2 Solution:} Data centric is one of the way in increasing the security in the Cloud Computing. If the data has gone into a public cloud, data security and governance control is transferred in a whole cloud provider. The best way of protecting the data in Cloud is by keeping the sensitive data secure using the data centric and file level encryption which works for both private and public Cloud Computing environment.
\item{RQ3 Solution:} The mitigating factors faced by the Cloud during the data security is water hole attack, compliance with data privacy laws in multiple geographies and so on. Coming to water hole attack done to Cloud during the security, this attack has 3 stages firstly the attacker does some reconnaissance and research on its target, in which employees of the targeted company who visits regularly. The second stage is insert an exploit into trusted sites. And finally takes advantage of their system vulnerabilities. The other method is Cloud environment is key and you must understand the respective data storage regulations. Most regulations include EU’s very restrictive regulations accept that this a good solution. These are the mitigating factors faced by the cloud during the data security.
\end{itemize}

\section{Discussion}

\subsection{Contribution}
This research provides to explore the improvement in security in cloud computing. Initially our focus was on Cloud Computing after gaining detailed knowledge in Cloud Computing we have found that the security is the major drawback in the cloud, so we have decided to work on the particular area that is about data security issue in cloud. So we have started our research on this topic. After the survey has done, different professionals have given their opinion on our questions we asked. RQ1 is based on the state of art framework related to data security, Our RQ2 is about the improvement in the quality of the data security and finally our RQ3 is about what is the mitigating factor facing regarding the data security by cloud.  

\subsection{Threats to validity}

\begin{itemize}

\item{Construct validity:} In our questionnaire the respondents may misinterpret some terms so this causes construct validity.
To avoid this, we kept our questionnaire simple.
\item{Internal validity:} There is a problem of drawing a wrong interface from the data collection this lead to chance in internal validity.
To avoid this, we used multiple methods of drawing the answers.
\item{External validity:} As we kept a survey on this there is a problem of getting less number of response this cause external validity threat.
To avoid this, we just increased the targeted population
\item{Conclusion validity:} Our conclusion may cause the conclusion validity due to ambiguous drawing in the answers.

\\To avoid this, we have spent enough time to draw the conclusions. 
\\

\section{Summary and Conclusions}

Cloud storage is one of the popular services provided by the cloud providers to store the customer data in a remote server. Even though the Cloud providers advertise that the stored information will be secure and intact, there are security attacks which lead to loss of data. To overcome the loss of data, the data security principles are implemented in different ways to protect the data [5]. It is a vital to take security and privacy into account when designing and using cloud computing services [6].
\paragraph{}
After this research we can conclude that improving the security issue in data can be done by data security method. Our results show the data security can be eradicated using the data centric protection. In further we can eradicate this type of security issues by CIA model also.
\paragraph{}
Due to limited amount of time we have collected only limited data that is only from 3 companies. In the future research we are planning to conduct more research on the methods which we are choosing.
