\section{Introduction}

Cloud Computing is one of the best emerging parameter in IT industry i.e. Cloud Computing is the latest trend in IT market. Users access their files and applications by accessing internet through cloud storage. This enable the user to utilize the set of resources of the server over the network. Cloud involves accessing the software applications. Cloud Computing also provides data storage and processing power over the Internet. Cloud Computing is not just a third party data warehouse. The data stored in the cloud may be frequently updated by the users, including insertion, deletion, modification, appending, reordering, etc.
The term cloud has been used historically as a metaphor for the Internet. This usage was originally Derived from its common depiction in network diagrams as an outline of a cloud, used to represent the transport of data across carrier backbones (which owned the cloud) to an endpoint location on the other side of the cloud.
There are three types of service layered models in Cloud Computing. Software as a service(SaaS), Platform as a Service(PaaS) and Infrastructure as a Service(IaaS). Customer obtain the facility to access & use an application or service that is hosted in the cloud for example Microsoft provides software as a service in Microsoft Office web application are accessible to office volume licensing customer and office web application is called SaaS. Customer obtain access to the platform by enabling them to organize their own software and application on cloud environment and control their application but they can’t manage servers and storage. It shares platform for customer software application configuration, testing and development of application this is called PaaS. The facility provided to the customer is to leave processing storage and other fundamental computing resources. The customer does not manage or control the basic cloud infrastructure but has control over Operating System, storage, deployed application is called IaaS.


\subsection{Background}

The main concern in Cloud Computing is the security issue, even though many researchers have been focused in this area they have not found the perfect solution yet. The impact of cloud services on the business sector is tremendous. With the increase in the end-users, there is an increasing growth in the number of Cloud Service Providers (CSP) as well. The CSP is a third party that maintains and manages information about another entity [3]. To know the security aspects and frameworks that are mitigating the issue which are present in current literature, we studied few Systematic Literature Review (SLR) and articles about security issues to gain knowledge in the present literature
As we studied about few SLR’s and articles regarding security issues in Cloud Computing, we gained some knowledge about what are the issues Cloud Computing is facing in security. By analyzing all the SLR’s we have found that the effective framework is missing to mitigate the data security model issues. So our purpose of this research is to provide improvement in quality of existing framework related to data security in Cloud Computing. 

\subsection{Objectives}

The main aim of these research is to improve the quality of the data security in cloud. That is how the data is going to be secured from malicious prosecution.
The main focus of this research is to make the readers gain the knowledge about quality improvement in  data security.

\subsection{Structure} 
The report is structured in the following way:  
Section III gives the motivation and background of this report. Section IV shows how the plan in conducting the research is done. Section V gives deals with how the research have been performed. Data extraction, analysis of data and interpretation of the results were discussed in Section VI. Previous knowledge with important novelties and their threats to validity in research is discussed in Section VII. Conclusion and summary of our work is explained in the final or last section.   


\nocite{*}
